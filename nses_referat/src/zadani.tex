\section{Zadání}
\begin{enumerate}
	\item Natrénujte:
	      \begin{enumerate}
		      \item jednovrstvou neuronovou síť
		      \item dvouvrstvou neuronovou síť
	      \end{enumerate}
	      Pro klasifikaci předmětů popsaných dvěma příznaky do 5 tříd.
	      Trénovací množiny jsou v souborech \texttt{tren\_data1.txt} a \texttt{tren\_data.txt}.
	      Každou síť natrénujte pro každou trénovací množinu zvlášť.

	      Každá řádka souborů s daty odpovídá jedné trénovací dvojici.
	      Na každém řádku jsou tři položky, které jsou od sebe odděleny mezerou.
	      První dvě položky reprezentují vstup sítě, třetí položka je informace o zařazení klasifikovaného předmětu do konkrétní třídy.
	\item Graficky znázorněte:
	      \begin{itemize}
		      \item průběh chyby v závislosti na trénovacích cyklech
		      \item oblast ve vstupním prostoru, jak výsledná síť klasifikuje vstupní body
	      \end{itemize}
	\item Funkční program osobně předveďte vyučujícímu.
\end{enumerate}

\emph{Upozornění:}
Vyučující může v bodě 3 požadovat, aby student změnil trénovací množinu, strukturu sítě, parametry trénování apod.
Dále může po studentovi požadovat vysvětlení, proč se síť v daném případě chová tak, jak se chová.
